% Created 2016-05-14 Sat 20:12
\documentclass{article}
\usepackage[utf8]{inputenc}
\usepackage[T1]{fontenc}
\usepackage{fixltx2e}
\usepackage{graphicx}
\usepackage{longtable}
\usepackage{float}
\usepackage{wrapfig}
\usepackage{rotating}
\usepackage[normalem]{ulem}
\usepackage{amsmath}
\usepackage{textcomp}
\usepackage{marvosym}
\usepackage{wasysym}
\usepackage{amssymb}
\usepackage{hyperref}
\tolerance=1000
\usepackage[a4paper, total={6in, 10in}]{geometry}
\usepackage[utf8]{inputenc}
\usepackage[english]{babel}
\usepackage{minted}
\usemintedstyle{emacs}
\renewcommand{\familydefault}{\rmdefault}
\usepackage[usenames, dvipsnames]{color}
\pagenumbering{arabic}
\usepackage{hyperref}
\hypersetup{colorlinks=true, linkcolor=blue, filecolor=magenta, urlcolor=cyan,}
\urlstyle{same}
\author{Sachin\thanks{psachin@redhat.com}}
\date{May 18, 2016}
\title{Red Hat OpenStack Platform - Swift}
\hypersetup{
  pdfkeywords={},
  pdfsubject={},
  pdfcreator={Emacs 25.0.93.1 (Org mode 8.2.10)}}
\begin{document}

\maketitle

\section{Theory}
\label{sec-1}
\begin{itemize}
\item Swift allows users to store unstructured data(objects) with
canonical names containing \emph{three} part:
\begin{itemize}
\item \texttt{/account}: Think \texttt{account} as a storage location and NOT user
account. \texttt{account} stores meta-data of that account plus list
of all containers in that account. \texttt{account} is analogous to
\texttt{/home} directory which may holds multiple users.

\item \texttt{/account/container}: Think of container as a root directory
of user(analogous to \texttt{/home/USER/}). Account can have many
containers with no limit.

\item \texttt{/account/container/object}: This is actual file. User may
start storing a single file(files are stored in container as
an object), or hierarchical data like \newline
\texttt{/photos/alaska/magic-bus/me.jpg} as an object. Swift stores
multiple copies of single object across physical locations to
ensure the data reliability and availability.
\end{itemize}

\item Remember that user do not have to know the actual location of the
data. In-fact he never knows. He always access the data in the
form of \texttt{/account/container/object}.
\end{itemize}

\section{Handy commands}
\label{sec-2}
\begin{itemize}
\item \emph{Note}: My Swift cluster is running on \texttt{192.168.8.80:8080}.
Following examples has \texttt{account} with the name \emph{wasteland},
\texttt{container} with the name \emph{keys} and file(\texttt{object}) with the name
\emph{mykey.pem}. My \texttt{AuthURL} is \emph{\url{http://192.168.8.80:8080/auth/v1.0}}
and \texttt{PubicURL/StorageURL} is
\emph{\url{http://192.168.8.80:8080/v1/AUTH_wasteland}}

\item Fetching info about Swift proxy. If you have access to Swift
server, \texttt{Pubic-URL/StorageURL} \& \texttt{account} can be fetched using
following command,
\begin{minted}[]{sh}
swift stat -v
#                     StorageURL: http://192.168.8.80:8080/v1/AUTH_wasteland
#                     Auth Token: AUTH_tk968b0ae7947640be874af6cd897a2b1e
#                        Account: AUTH_wasteland
#                     Containers: 0
#                        Objects: 0
#                          Bytes: 0
# Containers in policy "default": 1
#    Objects in policy "default": 0
#      Bytes in policy "default": 0
#                  Accept-Ranges: bytes
#                    X-Timestamp: 1463138224.81309
#                     X-Trans-Id: tx1fe3ecbeb9f04fdc92287-005735e92c
#                   Content-Type: text/plain; charset=utf-8
\end{minted}

\vskip15ex

\item New user can be added as follows,
\begin{minted}[]{sh}
# --- /etc/swift/proxy-server.conf ---
[filter:tempauth]
use = egg:swift#tempauth
# user_ACCOUNT_USER = PASSWORD [GROUP] <storage URL:8080>
user_wasteland_psachin = psachin .admin .reseller_admin

[app:proxy-server]
use = egg:swift#proxy
allow_account_management = true
account_autocreate = true
# --- File ends here ---

# Restart servers and Proxy
swift-init account start
swift-init container start
swift-init object start
swift-init proxy restart
\end{minted}
\item Managing container and object using \texttt{swift} command

Set following environmental variables
\begin{minted}[]{sh}
# ~/.profile
export ST_AUTH=http://192.168.8.80:8080/auth/v1.0
export ST_USER=wasteland:psachin
export ST_KEY=psachin
\end{minted}

Source the file before executing any command
\begin{minted}[]{sh}
source ~/.profile
\end{minted}

\emph{Most of the time, no configuration is needed, if Swift is
enabled during packstack. You can actually start from here.}
\begin{minted}[]{sh}
# --- Create container: 'keys' ---
swift post keys
# Verify/list containers
swift list
# --- Upload an object to container ---
# Create a file
echo "746c1c636cebe7a888fd77688dbfc252" > mykey.pem
# Upload object-'mykey.pem' to container-'keys'
swift upload keys mykey.pem
# Verify the object
swift list keys
# --- Download object ---
swift download keys mykey.pem
# Download object with different name
swift download keys mykey.pem -o mykey2.pem
\end{minted}
\item Managing container and object using APIs(\texttt{curl} command)
\begin{minted}[]{sh}
# --- Get token ---
# Set authURL and publicURL
export authURL="http://192.168.8.80:8080/auth/v1.0/"
export publicURL="http://192.168.8.80:8080/v1/AUTH_wasteland"

curl -v \
     -H "X-Auth-User: wasteland:psachin" \
     -H "X-Auth-Key: psachin" \
     $authURL

# *   Trying 192.168.8.80...
# * Connected to 192.168.8.80 (192.168.8.80) port 8080 (#0)
# > GET /auth/v1.0/ HTTP/1.1
# > Host: 192.168.8.80:8080
# > User-Agent: curl/7.43.0
# > Accept: */*
# > X-Auth-User: wasteland:psachin
# > X-Auth-Key: psachin
# >
# < HTTP/1.1 200 OK
# < X-Storage-Url: http://192.168.8.80:8080/v1/AUTH_wasteland
# < X-Auth-Token-Expires: 82975
# < X-Auth-Token: AUTH_tk968b0ae7947640be874af6cd897a2b1e
# < Content-Type: text/html; charset=UTF-8
# < X-Storage-Token: AUTH_tk968b0ae7947640be874af6cd897a2b1e
# < Content-Length: 0
# < X-Trans-Id: tx9c1bef9065754dd9b68ec-005735c49d
# < Date: Fri, 13 May 2016 12:12:13 GMT
# <
# * Connection #0 to host 192.168.8.80 left intact

export token="AUTH_tk968b0ae7947640be874af6cd897a2b1e"

# Verify account access
curl -v \
     -H "X-Storage-Token: $token" \
     $publicURL

# *   Trying 192.168.8.80...
# * Connected to 192.168.8.80 (192.168.8.80) port 8080 (#0)
# > GET /v1/AUTH_wasteland HTTP/1.1
# > Host: 192.168.8.80:8080
# > User-Agent: curl/7.43.0
# > Accept: */*
# > X-Storage-Token: AUTH_tk968b0ae7947640be874af6cd897a2b1e
# >
# < HTTP/1.1 204 No Content
# < Content-Length: 0
# < Accept-Ranges: bytes
# < X-Account-Object-Count: 0
# < X-Account-Storage-Policy-Default-Bytes-Used: 0
# < X-Account-Storage-Policy-Default-Object-Count: 0
# < X-Timestamp: 1463138224.81309
# < X-Account-Bytes-Used: 0
# < X-Account-Container-Count: 0
# < Content-Type: text/plain; charset=utf-8
# < X-Account-Storage-Policy-Default-Container-Count: 0
# < X-Trans-Id: tx95142c218202459399c88-005735cac1
# < Date: Fri, 13 May 2016 12:38:25 GMT
# <
# * Connection #0 to host 192.168.8.80 left intact

# --- Create a container: 'keys' ---
curl -v \
     -H "X-Storage-Token: $token" \
     -X PUT $publicURL/keys

# *   Trying 192.168.8.80...
# * Connected to 192.168.8.80 (192.168.8.80) port 8080 (#0)
# > PUT /v1/AUTH_wasteland/keys HTTP/1.1
# > Host: 192.168.8.80:8080
# > User-Agent: curl/7.43.0
# > Accept: */*
# > X-Storage-Token: AUTH_tk968b0ae7947640be874af6cd897a2b1e
# >
# < HTTP/1.1 201 Created
# < Content-Length: 0
# < Content-Type: text/html; charset=UTF-8
# < X-Trans-Id: tx39b7aee463b64127adfe2-005735cb92
# < Date: Fri, 13 May 2016 12:41:54 GMT
# <
# * Connection #0 to host 192.168.8.80 left intact

# Verify container
curl -v \
     -H "X-Storage-Token: $token" \
     -X GET $publicURL/keys

# *   Trying 192.168.8.80...
# * Connected to 192.168.8.80 (192.168.8.80) port 8080 (#0)
# > GET /v1/AUTH_wasteland/keys HTTP/1.1
# > Host: 192.168.8.80:8080
# > User-Agent: curl/7.43.0
# > Accept: */*
# > X-Storage-Token: AUTH_tk968b0ae7947640be874af6cd897a2b1e
# >
# < HTTP/1.1 204 No Content
# < Content-Length: 0
# < X-Container-Object-Count: 0
# < Accept-Ranges: bytes
# < X-Storage-Policy: default
# < X-Container-Bytes-Used: 0
# < X-Timestamp: 1463138224.83257
# < Content-Type: text/html; charset=UTF-8
# < X-Trans-Id: tx05408e3d41c246ea930f5-005735cc21
# < Date: Fri, 13 May 2016 12:44:17 GMT
# <
# * Connection #0 to host 192.168.8.80 left intact

# --- Upload object to container ---
# Create a file
echo "746c1c636cebe7a888fd77688dbfc252" > mykey.pem

# Upload object-'mykey.pem' to container-'keys'
curl -v \
     -H "X-Storage-Token: $token" \
     -X PUT $publicURL/keys/mykey.pem -T mykey.pem

# *   Trying 192.168.8.80...
# * Connected to 192.168.8.80 (192.168.8.80) port 8080 (#0)
# > PUT /v1/AUTH_wasteland/keys/mykey.pem HTTP/1.1
# > Host: 192.168.8.80:8080
# > User-Agent: curl/7.43.0
# > Accept: */*
# > X-Storage-Token: AUTH_tk968b0ae7947640be874af6cd897a2b1e
# > Content-Length: 43
# > Expect: 100-continue
# >
# < HTTP/1.1 100 Continue
# * We are completely uploaded and fine
# < HTTP/1.1 201 Created
# < Last-Modified: Fri, 13 May 2016 12:53:00 GMT
# < Content-Length: 0
# < Etag: 640ebd176639fb6ef9a3227770ee7b17
# < Content-Type: text/html; charset=UTF-8
# < X-Trans-Id: txf33923d6fbfe4523b4451-005735ce2b
# < Date: Fri, 13 May 2016 12:52:59 GMT
# <
# * Connection #0 to host 192.168.8.80 left intact

# Download an object
curl -v \
     -H "X-Storage-Token: $token" \
     -X GET $publicURL/keys/mykey.pem > mykey.pem

# *   Trying 192.168.8.80...
#   % Total    % Received % Xferd  Average Speed   Time    Time     Time  Current
#                                  Dload  Upload   Total   Spent    Left  Speed
    #   0     0    0     0    0     0      0      0 --:--:-- --:--:-- --:--:-- 0* \
#                          Connected to 192.168.8.80 (192.168.8.80) port 8080 (#0)
# > GET /v1/AUTH_wasteland/keys/mykey.pem HTTP/1.1
# > Host: 192.168.8.80:8080
# > User-Agent: curl/7.43.0
# > Accept: */*
# > X-Storage-Token: AUTH_tk968b0ae7947640be874af6cd897a2b1e
# >
# < HTTP/1.1 200 OK
# < Content-Length: 43
# < Accept-Ranges: bytes
# < Last-Modified: Fri, 13 May 2016 12:53:00 GMT
# < Etag: 640ebd176639fb6ef9a3227770ee7b17
# < X-Timestamp: 1463143979.89953
# < Content-Type: application/octet-stream
# < X-Trans-Id: tx6b14a272331b4bc6937db-005735cef1
# < Date: Fri, 13 May 2016 12:56:17 GMT
# <
# { [43 bytes data]
# 100    43  100    43    0     0   2748      0 --:--:-- --:--:-- --:--:--  2866
# * Connection #0 to host 192.168.8.80 left intact
\end{minted}

\item Object versioning
\begin{minted}[]{sh}
# To obtain Storage URL and Auth-Token
swift stat -v

# Get statistics of container and/or object
swift stat [container]
swift stat [container] [object]

# Retrive capability of proxy
swift capabilities

# List container's details(Similar to `ls -lh`)
swift list --lh [container]

# 'archive' container to hold 'current' container's object versions
swift post archive
swift post current -H "X-Versions-Location: archive"

# May also define content length at the time of creating a container
swift post archive -H "content-length: 0"
swift post current -H "content-length: 0" -H "X-Versions-Location: archive"

# And also specify ACL(World readable) during container creation
swift post -r ".r:*" archive -H "content-length: 0"
swift post -r ".r:*" current -H "content-length: 0" -H "X-Versions-Location: archive"
\end{minted}
\end{itemize}

\section{Additional notes}
\label{sec-3}
\emph{Swift} consistency processes:
\begin{itemize}
\item \emph{Auditor}: Will walk through the storage nodes, read the data and
the checksum, ensure the checksum matched with the database
checksum. If the checksum didn't match, the data is moved to the
Quarantine.
\item \emph{Replicator}: The replicator, will also scan each drive and
ensures that the replicas of data is stored where is supposed to
live. If it does not finds the data in that place(may be the
data, due to corruption was moved to Quarantine), it will push
the data to that place.
\end{itemize}

\section{Slides notes}
\label{sec-4}

\begin{itemize}
\item Multiple HDD, where is my data store?
\item HDD failure
\item Storage problem

\item Ownership of your data
\item Access to data, HTTP, FTP, ReST
\begin{itemize}
\item Mobile, Laptop..
\end{itemize}

\item Swift
\begin{itemize}
\item loosely tied to storage media
\item Scalable
\item Direct client access
\end{itemize}
\end{itemize}


\begin{itemize}
\item Terminology
\begin{itemize}
\item Proxy: provides API access/ Coordinates requests to storage
servers
\item Account: user namespace
\item Container: User defined segment of an account(root directory)
\item Object: Actual data
\end{itemize}

\item Flow
Proxy request -> Storage nodes(account, container, obj)

\item Data placement
\begin{itemize}
\item triple replication by default(as unique as possible)
\item Show Region/Zone pic
\end{itemize}

\item Drive failures
\begin{enumerate}
\item Umount failing drive
\item Replicate/rebalance data
\end{enumerate}

\item Server failures
\begin{enumerate}
\item Network, Power
\item New data that is to be written will be placed elsewhere within a
cluster/server
\item Rebalancing happens
\end{enumerate}

\item Currupt data
\begin{enumerate}
\item Stores checksum of the data with data itself
\item Matches checksum of data periodically
\begin{itemize}
\item If checksum doesnt match, the object is quarantined and the
replication process rebalances the data/object
\end{itemize}
\end{enumerate}

\item Storage policies
\begin{itemize}
\item Decide where you want to store data
\begin{itemize}
\item Between swift clusters
\item Subset of hardware
\end{itemize}
\item Erasure coding <- Data availability policies
\begin{itemize}
\item Based in frequency of access
\end{itemize}
\end{itemize}

\item ACLs
\end{itemize}

\section{Links}
\label{sec-5}
\begin{itemize}
\item \url{https://gitlab.cee.redhat.com/psachin/bootcamp}
\item HTML version of this\footnote[1]{Made with Love, LaTeX \& GNU Emacs} doc:
\url{https://gitlab.cee.redhat.com/psachin/bootcamp/blob/master/2016/scripts/notes.org}
\item Slides: \url{https://redhat.slides.com/psachin/rhosp-swift-2016}
\end{itemize}
% Emacs 25.0.93.1 (Org mode 8.2.10)
\end{document}
